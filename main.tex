\documentclass{article}

\usepackage{ifxetex}
\ifxetex
\usepackage{fontspec}
\else
\usepackage[T1]{fontenc}
\usepackage[utf8]{inputenc}
\usepackage{lmodern}
\fi

\title{MATH UA 120 Definitions}
\date{}

\usepackage{amsthm}
\usepackage{amssymb}
\usepackage{upgreek}
\usepackage{amsmath}

\usepackage{tikz}
\usetikzlibrary{calc}
\usetikzlibrary{positioning}
\usetikzlibrary{patterns}

\usepackage{hyperref}

\begin{document}
	\maketitle
	
	\begin{enumerate}
		\item Divisible: Let $a,b$ be integers.
		We say that $b$ divides $a$ if
		
		\[\exists c\in\mathbb{Z}\:\: \text{s.t.} \:\: bc=a\]
		
		\item Even: An integer $a$ is called even provided $2|a$.
		
		\item Odd: An integer $b$ is called odd if
		
		\[\exists a\in\mathbb{Z} \:\: \text{s.t.} \:\: b=2a+1\]
		
		\item Prime: An integer $p$ is prime if $p>1$ and the only positive divisors of $p$ are $1$ and $p$.
		
		\item Composite: An integer $a$ is composite if $a>0$ and there is an integer $b$ s.t. $1<b<a$ and $b|a$.
		
		\item Perfect integer: An integer $a$ is a perfect integer provided
		
		\[a=\sum_{n=1}^k x,\:\: x_n|a\]
		
		\item List: A list is an ordered collection of items.
		
		\item Multiplication Principle: The Multiplication Principle states that for 2-tuple with $m$ choices for the first element and $n$ choices for the second element, there are $mn$ possible lists that may be formed.
		
		\item Falling Factorial: Let $n,k\in\mathbb{N}$.
		Falling factorial, denoted $(n)_k$, is calculated as
		
		\[(n)_k=n(n-1)(n-2)\dots(n-k+1)\]
		
		\item Factorial: Let $n,k\in\mathbb{N}$.
		Factorial is equivalent to falling factorial when $n=k$.
		We denote factorial as $n!$
		
		\item Set: A set is a repetition-free, unordered collection of objects.
		
		\item Subset: A set $A$ is called a subset of $B$ if for every $x\in A$, $x\in B$.
		We denote this as $A\subseteq B$.
		
		\item Superset: A set $A$ is called a superset of $B$ if for every $x\in B$, $x\in A$.
		We denote this as $A\supseteq B$.
		
		\item Strict or Proper Subset: A set $A$ is called strict, or proper, subset of $B$ if $A \subseteq B$, $A\neq B$.
		This is denoted $A\subset B$
		
		\item Power set: If $A$ is a set, the power set of $A$ is the set of all subsets of $A$.
		The power set of $A$ is denoted $2^A$.
		
		\item Union: The union of two sets $A$, $B$ is the set of all elements that are in $A$ or $B$.
		The union of $A$ and $B$ is denoted $A\cup B$.
		
		\item Intersection: The intersection of two sets $A$, $B$ is the set of all elements that are in $A$ and $B$.
		The intersection of $A$ and $B$ is denoted $A\cap B$.
		
		\item Cardinality: The cardinality of a set $A$ is the number of elements in $A$.
		We denote this as $|A|$.\\
		
		Note that given two sets $A$ and $B$,
		
		\[|A|+|B|=|A\cap B|+|A\cup B|\]
		
		\item Disjoint: Two sets $A$ and $B$ are called disjoint provided $A\cap B=\emptyset$.
		
		\item Pairwise Disjoint: Let $A_1,A_2,\dots,A_n$ be a collection of sets.
		These sets are called pairwise disjoint provided $A_i\cap A_j=\emptyset$ whenever $i\neq j$.
		
		\item Addition Principle: Let $A$ and $B$ be finite sets.
		If $A$ and $B$ are disjoint, then
		
		\[|A\cup B|=|A|+|B|\]
		
		\item Set Difference: Let $A$ and $B$ be sets.
		The set difference, denoted $A-B$, is the set of all elements of $A$ that are not in $B$.
		
		\item Symmetric Difference: Given two sets $A$ and $B$, the symmetric difference, denoted $A\Delta B$, is
		\[(A-B)\cup (B-A)\]
		
		\item Cartesian Product: Given two sets $A$ and $B$, their Cartesian product, denoted $A\times B$, is the set of all 2-tuples formed by taking an element from A together with an element from B.
		
		\item Universe: A universe is the set of all objects under consideration for a certain context
		
		\item Set Complement: Given a set $A$, the complement of $A$, denoted $\bar A$, is the set of all elements of the universe that are not in $A$.
		
		\item Relation: A relation is a set of ordered pairs.
		
		\item Relation On: Let $R$ be a relation and $A$ a set.
		We say $R$ is a relation on $A$ provided $R\subseteq A\times A$.
		
		\item Relation From: Let $R$ be a relation and $A$, $B$ sets.
		We say $R$ is a relation from $A$ to $B$ provided $R\subseteq A\times B$.
		
		\item Inverse Relation: Let $R$ be a relation.
		The inverse of $R$, denoted $R^{-1}$, is the relation formed by reversing the order of all ordered pairs in $R$.
		
		\item Reflexive: A relation $R$ on $A$ is reflexive if for all $x\in A$, $x$ $R$ $x$.
		
		\item Irreflexive: A relation $R$ on $A$ is irreflexive if for all $x\in A$, $x$ $\not R$ $x$.
		
		\item Symmetric: A relation $R$ on $A$ is symmetric if for all $x,y\in A$,
		
		$x$ $R$ $y\implies y$ $R$ $x$.
		
		\item Antisymmetric: A relation $R$ on $A$ is antisymmetric if for all $x,y\in A$,
		
		($x$ $R$ $y\wedge y$ $R$ $x$) $\implies x=y$.
		
		\item Transitive: A relation $R$ on $A$ is transitive if for all $x,y,z\in A$,
		
		($x$ $R$ $y\wedge y$ $R$ $z$) $\implies x$ $R$ $z$.
		
		\item Equivalence Relation: A relation is called an equivalence relation if it is reflexive, symmetric, and transitive.
		
		\item Congruence Modulo n: Let $n$ be a positive integer.
		Integers $x$ and $y$ are congruent modulo n if $n|(x-y)$.
		This is denoted $x \equiv y$ $(\text{mod}$ $n)$.\\
		
		Note that the relation is-congruent modulo n-to is an equivalence relation.
		
		\item Equivalence class: Let $R$ be an equivalence relation on a set $A$ and let $a\in A$.
		The equivalence class of $a$, denoted $[a]$, is the set of all elements related by $R$ to $a$.
		
		\item Partition: Let $A$ be a set.
		A partition on $A$ is a set of nonempty, pairwise disjoint sets whose union is $A$.
		
		\item Binomial coefficient: Let $n,k\in\mathbb{N}$.
		The symbol ${n\choose k}$ denotes the number of $k$ element subsets of an $n$ element set.\\
		
		Note that ${n\choose k}=\frac{n!}{k!(n-k)!}$
		
		\item Function or Mapping: Let $A$ and $B$ be sets with $a\in A$ and $b,c\in B$.
		A function $f$ is a relation on $A\times B$ provided $(a,b)\in f$ and $(a,c)\in f$ imply $b=c$.
		We may denote this function as $f: A\to B$.
		
		\item Domain: the domain of a function is the set of all possible first elements in the ordered pair of the function.
		In other words,
		
		\[\text{dom} f= \{a: \exists b, (a,b)\in f\}\]
		
		\item Image: the image of a function is the set of all possible second elements in the ordered pair of the function.
		In other words,
		
		\[\text{im} f= \{b: \exists a, (a,b)\in f\}\]
		
		\item Co-domain: The co-domain of a function is a superset of the image.
		
		\item Injective or One-to-One: A function $f$ is injective provided whenever $(x,b),(y,b)\in f$, $x=y$.
		An alternate definition is that $f$ is injective iff the inverse relation $f^{-1}$ is a function.
		
		\item Surjective or Onto: Let $f:A\to B$.
		A function is surjective when for every $b\in B, \exists a\in A$ s.t. $f(a)=b$.
		An alternate definition is that $f$ is surjective iff the image of $f$ equals the co-domain of $f$.
		
		\item Bijective: A function is bijective when it is both surjective and injective.
		An alternate definition is $f$ is bijective iff $f:A\to B$, then $f^{-1}:B\to A$.
		
		\item Bijection: A bijective function is called a bijection.
		
		\item Pigeonhole Principle: Let $A$ and $B$ be finite sets with $f:A\to B$.
		If $|A|>|B|$, then $f$ is not one-to-one.
		If $|B|>|A|$, then $f$ is not onto.
		
		\item Erdos-Szekeres: Let $n$ be a positive integer.
		Every sequence of $n^2+1$ distinct integers must contain a monotone subsequence of length $n+1$.
		
		\item Cantor's Theorem: $|A|<2^{|A|}$ for all non-empty sets $A$.
		
		\item Composition of Functions: Let $A,B,C$ be sets and let $f:A\to B$ and $g:B\to C$.
		Then $g\circ f(a)=g[f(a)]$.
		
		\item Sample Space: A sample space is a pair $(S,P)$ where $S$ is a finite, nonempty set and $P$ is a function
		
		\[P:S\to\mathbb{R}\: \text{s.t.}\: P(s)\geq 0 \: \text{for all}\: s\in S \:\text{and}\: \sum_{s\in S}P(s)=1\]
		
		\item Event: An event $A$ is a subset of $S$.
		The probability of an event, denoted $P(A)$, is given by
		
		\[P(A)=\sum_{a\in A}P(a)\]
		
		\item Conditional probability: Let $A$ and $B$ be events in a sample space $(S,P)$ and suppose $P(B)\neq 0$.
		The conditional probability of $A$ given $B$ is denoted $P(A|B)$ and is given by
		
		\[P(A|B)=\frac{P(A\cap B)}{P(B)}\]
		
		\item Independence: Two events are called independent if these equivalent conditions are met:
		
		\[P(A|B)=P(A)\:\: \text{or} \:\: P(B|A)=P(B) \:\: \text{or} \:\: P(A\cap B)=P(A)P(B)\]
		
		\item Independent Repeated Trials: Let $(S,P)$ be a sample space and $n$ a positive integer.
		Let $S^n$ denote the set of all length $n$ lists of elements in $S$.
		Then $(S^n,P)$ is the $n$-fold repeated-trial sample space in which:
		
		\[P[(s_1,s_2,\dots,s_n)]=P(s_1)P(s_2)\dotsP(s_n)\]
		
		\item Random variable: A random variable is a function defined on a probability space;
		that is, if $(S,P)$ is a sample space, then a random variable is a function $X:S\to V$ for some set $V$.
		
		\item Independent Random Variables: Let $(S,P)$ be a sample space and let $X$ and $Y$ be random variables defined on $(S,P)$.
		We say that $X$ and $Y$ are independent if, for all $a,b$,
		
		\[P(X=a\: \text{and} \: Y=b)=P(X=a)P(Y=b)\]
		
		\item Binomial Random Variable: Let $X$ be a random variable and $A$ an event with $P(A)=p$.
		If we conduct $n$ independent repeated trials, the probability that $A$ happens $h$ times is given by:
		
		\[P(X=h)= {n \choose h} p^h(1-p)^{n-h}\]
		
		\item Expected Value: Let $X$ be a real-valued random variable defined on a sample space $(S,P)$.
		The expectation of $X$, denoted $E(X)$ or $\mu$, is
		
		\[E(X)=\mu=\sum_{s\in S}X(s)P(s)=\sum_{a\in\mathbb{R}}aP(X=a)\]
		
		\item Variance: Let $X$ be a real-valued random variable on a sample space $(S,P)$.
		Let $u=E(X)$.
		The variance of $X$ is
		
		\[\text{Var}(X)=E[(X-\mu)^2]=E[X^2]-E[X]^2\]
		
		\item Division: Let $a,b\in\mathbb{Z}$ with $b>0$.
		Then there exist unique integers $q$ and $r$ s.t.
		
		\[a=qb+r \: \text{and} \: 0\leq r<b\]
		
		\item Div and Mod: Let $a,b\in\mathbb{Z}$ with $b>0$.
		We know that there are two unique integers $q,r$ with $a=qb+r$ and $0\leq r<b$.
		Then
		
		\[a \:\text{div}\:b=q,\: a \:\text{mod}\: b=r\]
		
		Note that $a\equiv b (\mod n) \Longleftrightarrow a\mod n=b\mod n$.
		
		\item Common Division: Let $a,b\in\mathbb{Z}$.
		We call an integer $d$ a common divisor of $a,b$ provided $d|a$ and $d|b$.
		
		\item Greatest Common Divisor: Let $a,b\in\mathbb{Z}$.
		We call an integer $d$ the greatest common divisor of $a$ and $b$ provided $d$ is a common divisor of $a$ and $b$ and if $e$ is a common divisor of $a$ and $b$, then $e\leq d$.
		We denote the greatest common divisor of $a$ and $b$ as $\gcd(a,b)$.\\
		
		Note that if $a$ and $b$ are non-zero integers, the smallest positive integer of the form $ax+by$, for integers $x,y$, is $\gcd(a,b)$.
		
		\item Euler's Algorithm: Euler's Algorithm states that if $a,b$ are positive integers and $c=a\: \text{mod} \: b$, then
		
		\[\gcd(a,b)=\gcd(b, c)\]
		
		\item Graph: A graph is a pair $G=(V,E)$, where $V$ is a nonempty finite set and $E$ is a set of two-element subsets of $V$.
		Elements of $V$ are called vertices, nodes, or points;
		elements of $E$ are called edges, arcs, links, or lines.
		
		\item Vertex and Edge Sets: the set of all vertices of a graph $G$ is denoted $V(G)$; the set of all edges of a graph $G$ is denoted $E(G)$.
		
		\item Adjacency: Let $G=(V,E)$ be a graph and let $u,v\in V$.
		We say that $u$ is adjacent to (or joined to) $v$ if $\{u,v\}\in E$.
		We denote this as $u\sim v$.
		
		\item Incidence: Let $v$ be a vertex and an endpoint of edge $e$; in other words, $v\in e$.
		Then we say $v$ is incident on $e$.
		
		\item Neighbor: Two vertices that are adjacent are called neighbors.
		
		\item Neighborhood: The set of all neighbors of a vertex $v$ is called the neighborhood of $v$ and denoted $N(v)$.
		
		\item Degree: Let $G=(V,E)$ be a graph and let $v\in V$.
		The degree of $v$ is the number of edges with which $v$ is incident.
		We denote the degree of $v$ as $d_G(v)$ or $d(v)$ if there is no risk of confusion.\\
		
		Note that the sum of the degrees of all vertices in a graph $G$ is twice the number of edges.\\
		
		Also note that the maximum degree of a vertex in $G$ is denoted $\Delta(G)$, while the minimum degree is denoted $\delta(G)$.
		
		\item Regular Graph: A graph $G$ is regular if all vertices in $G$ have the same degree.
		If a graph is regular and all is vertices have degree $r$, the graph is called $r-$regular.
		
		\item Order and Size: Let $G(V,E)$ be a graph.
		The order of $G$ is $|V|$.
		The size of $G$ is $|E|$.
		Customarily, we set $n=|V|$ and $m=|E|$.
		However, some authors use $\nu(G)=|V(G)|$, $\upvarepsilon(G)=|E(G)|$.\\
		
		Note that in the latter denotation, we think of $\nu$ and $\upvarepsilon$ as functions that map from a graph $G$ to the natural numbers.
		
		\item Complete Graph: Let $G$ be a graph.
		If all pairs of distinct vertices are adjacent in $G$, we call $G$ complete.
		A complete graph on $n$ vertices is denoted $K_n$.
		
		\item Edgeless Graph: Let $G$ be a graph.
		If $E(G)$ is empty, we call $G$ edgeless.
		An edgeless graph on $n$ vertices is denoted $E_n$.
		
		\item Subgraph: Let $G$ and $H$ be graphs.
		We call $G$ a subgraph of $H$ provided $V(G)\subseteq V(H)$ and $E(G)\subseteq E(H)$.
		
		\item Spanning Subgraph: Let $G$ and $H$ be graphs.
		We call $G$ a spanning subgraph of $H$ provided $G$ is a subgraph of $H$ and $V(G)=V(H)$.\\
		
		In other words, $G$ is $H$ with at least one edge removed.
		If this edge is named $e$, then we denote this as $G=H-e$.
		
		\item Induced Subgraph: Let $H$ be a graph and let $A\subseteq V(H)$.
		The subgraph of $H$ induced on $A$ is the graph $H[A]$ defined by
		
		\[V(H[A])=A, \: E(H[A])=\{\{x,y\}\in E(H): x,y\in A\}\]
		
		\item Clique, Clique Number: Let $G$ be a graph.
		A subset of vertices $S\subseteq V(G)$ is called a clique provided any two distinct vertices in $S$ are adjacent.
		In other words, a set $S\subseteq V(G)$ is called a clique provided $G[S]$ is a complete graph.\\
		
		The clique number of $G$ is the size of a largest clique;
		it is denoted $\omega (G)$.
		
		\item Independent Set, Independence Number: Let $G$ be a graph.
		A subset of vertices $S\subseteq V(G)$ is called an independent set provided no two vertices in $S$ are adjacent.
		In other words, a set $S\subseteq V(G)$ is called an independent set provided $G[S]$ is an edgeless graph.\\
		
		The independence number of $G$ is the size of a largest independent set;
		it is denoted $\alpha (G)$.
		
		\item Complement: Let $G$ be a graph.
		The complement of $G$, denoted $\bar G$, is defined by
		
		\[V(\bar G)=V(G)\: \text{and}\: E(\bar G)=\{\{x,y\}: x,y \in V(G), x\neq y, \{x,y\}\not\in E(G)\}\]
		
		Note that $\omega(G)=\alpha(\bar G)$ and $\alpha(G)=\omega(\bar G)$.
		
		\item Ramsey's Theorem: either $G$ or its complement, $\bar G$, has a "large" clique.
		In a special case, if $G$ is a graph with at least order $6$, then $\omega(G)\geq 3$ or $\omega(\bar G)\geq 3$.
		
		\item Walk: Let $G=(V,E)$ be a graph.
		A walk in $G$ is a list of vertices, with each vertex adjacent to the next;
		that is, $W=v_0\sim v_1\sim\dots\sim v_\ell$.
		If $u,v\in V(G)$ and there is a walk whose first vertex is $u$ and last vertex is $v$, then we say there is a $(u,v)-$walk in $G$.\\
		
		The length of the walk is $\ell$.
		However, there are actually $\ell +1$ vertices on the walk.
		
		\item Concatenation: Let $G$ be a graph and let $W_1$, $W_2$ be walks in $G$ with $W_1=v_0\sim v_1\sim \dots\sim v_\ell$ and $W_2=w_0\sim w_1\sim \dots\sim w_k$.
		Suppose $v_\ell=w_0$.
		Their concatenation, denoted $W_1+W_2$, is the walk
		
		\[W_1+W_2=v_0\sim v_1\sim\dots\sim(v_\ell=w_0)\sim w_1\sim\dots\sim w_k\]
		
		Note that $W_1+W_2$ has length $n+k$ (but with $n+k+2$ vertices on the walk!)
		
		\item Path: A path in a graph $G$ is a walk in $G$ in which no vertex is repeated.
		If $u,v\in V(G)$ and there is a path whose first vertex is $u$ and last vertex is $v$, then we say there is a $(u,v)-$path in $G$. \\
		
		Note that if there is a $(u,v)-$walk in $G$, the shortest $(u,v)-$ walk is a $(u,v)-$path.
		
		\item Path (graph): A path is a graph with vertex set $V=\{v_1,v_2,\dotsv_n\}$ and edge set $E=\{\{v_i,v_{i+1}\}:1\leq i<n\}$.\\
		
		A path on $n$ vertices is denoted $P_n$.
		
		\item Hamiltonian Path: Let $G$ be a graph.
		A path in $G$ that contains all the vertices of $G$ is called a Hamiltonian path.
		
		\item Connected to: let $G$ be a graph and let $u,v\in V(G)$.
		If a $(u,v)-$path exists, then we say $u$ is connected to $v$.\\
		
		Note that the relation is-connected-to is an equivalence relation.
		
		\item Component: A component of $G$ is a subgraph of $G$ induced on an equivalence class of the is-connected-to relation on $V(G)$.
		
		\item Connected: A graph is called connected provided each pair of vertices in the graph is connected by a path.
		Equivalently, a graph is called connected if it only has one component.
		
		\item Cut Vertex: Let $G$ be a graph.
		A vertex $v\in V(G)$ is called a cut vertex of $G$ provided $G-v$ has more components than $G$.
		
		\item Cut Edge: Let $G$ be a graph.
		An edge $v\in V(G)$ is called a cut edge of $G$ provided $G-e$ has more components than $G$.
		
		\item Cycle (walk): A cycle is a walk of length at least three in which the first and last vertex are the same, but no other vertices are repeated.
		
		\item Cycle (graph): A cycle is a graph or subgraph consisting of the vertices and edges of a cycle (walk).
		In other words, a cycle is a graph of the form $G=(v,E)$ where
		
		\begin{gather*}
			V=\{v_1,v_2,\dots,v_n\} \:\: \\
			E=\{\{v_1, v_2\}, \{v_3, v_4\}\dots,\{v_{n-1}, v_n\},\{v_n, v_1\}\}\\
		\end{gather*}
		
		A graph with $n$ vertices consisting solely of a cycle is denoted $C_n$.
		
		\item Hamiltonian Cycle: Let $G$ be a graph.
		A cycle of $G$ that contains all the vertices in $G$ is called a Hamiltonian cycle.
		
		\item Forest or Acyclic Graph: Let $G$ be a graph.
		If $G$ contains no cycles, then we call $G$ a forest.
		Alternatively, we call $G$ acyclic.
		
		\item Tree: A tree is a connected, acyclic graph.
		In other words, a tree is a connected forest.
		We denote a tree with $n$ vertices as $T_n$.\\
		
		Note that a single isolated vertex ($K_1$ or $E_1$) is a tree.\\
		
		There are also several alternate definitions of a tree:
		
		\begin{enumerate}
			\item A graph $G$ is a tree iff for any two vertices $u,v\in V(G)$, there is a unique $(u,v)-$path.
			\item A connected graph $G$ is a tree iff every edge of $G$ is a cut edge.
			\item A connected graph $G$ with $n\geq 1$ vertices is a tree iff $G$ has exactly $n-1$ edges.
		\end{enumerate}
		
		\item Leaf: A leaf of a graph is a vertex of degree 1.
		
		\item Spanning Tree: Let $G$ be a graph.
		A spanning tree of $G$ is a spanning subgraph of $G$ that is a tree.
	
	\end{enumerate}

\end{document}
